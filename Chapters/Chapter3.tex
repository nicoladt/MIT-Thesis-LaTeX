% Chapter 3

\chapter{Related Research} % Main chapter title

\label{Chapter3} % For referencing the chapter elsewhere, use \ref{Chapter1} 

\lhead{Chapter 3. \emph{Related Research}} % This is for the header on each page - perhaps a shortened title

%----------------------------------------------------------------------------------------
The Concise Oxford English Dictionary describes an annotation as an explanatory note added to a book or document \citep{OxfordDict}. Haslhofer et al. \citep{LEMO} extend this and state that an annotation can be seen as "a remark, explanation or interpretation added to the original document". According to Ovsiannikov et al. \citep{Ovsiannikov} annotations may take the form of written notes, a symbol, a drawing or, in the digital space, a multimedia clip. 

Analogue annotations and marginalia in books and other hardcopy documents have a long tradition \citep{LEMO} and come in a variety of formats (some more formal than others) including handwritten notes in margins, printed margin notes in textbooks and Post-it notes stuck on to content. More recently, with the inevitable shift towards digital reading, annotations have become possible in desktop software suites, on e-reading devices such as Amazon's Kindle, and on the World Wide Web. 

Much research has been done into the different types of annotations that exist \citep{Marshall2000} \citep{Marshall2004}, the workflows by which they are created, and their purpose and usage  particularly in the digital realm \citep{Agosti} \citep{Ovsiannikov}. Agosti et al. \citep{Agosti} name three major uses for annotations: to create new information resources, to interpret existing ones, to access resources in new ways, and to support the effective use of resources.  Arko et al. \citep{Arko} state that annotations also allow for community engagement and the capturing of "ephemeral information" that would otherwise be "lost in transient media like conversation" or emails. This is applicable to analogue annotations, but is particularly true of digital annotations, which can so easily be shared, viewed and processed online.


While desktop software to make and read annotations (or "notes" or "comments") has been around for years (e.g. MSOffice \citep{MSOffice} for Microsoft documents, and Adobe Reader \citep{Adobe} for PDFs), a multitude of new technologies are becoming available due to increasing interconnectivity and new storage options given to us by the Web. Examples of online annotation software include Google Drive \citep{GDrive} (for documents and spreadsheets), A.nnotate.com \citep{AnnotateCom} and AnnotateIt \citep{AnnotateIt}.

In terms of accessing information stored on the Web, faceted searches and dynamic queries of databases are becoming increasingly common. Faceted search and navigation systems similar to those used by Amazon.com \citep{Amazon} or eBay \citep{eBay} are well documented \citep{Hearst} \citep{Yitzhak}, as are the ways in which users interact with these kind of search interfaces \citep{Kules}. Similarly, dynamic queries of databases provide a powerful method for browsing and searching, enabling users not only to explore databases of information but also to efficiently filter out unnecessary information and locate a specific piece of information \cite{ShneidermanDynamic} \citep{PlaisantQueryPreview}. Information visualization is also key component of designing interfaces that present sets of database entries to users in meaningful and intuitive visual representations \citep{Card}. 

As online annotation possibilities have expanded, several frameworks have been developed to try and standardise the ways in which annotations can me made, stored and manipulated, particularly on the Semantic Web \citep{Berners-Lee} \citep{Uren}. Notable examples of such frameworks include Annotea \citep{kahan2002annotea}, CREAM \citep{handschuh2002authoring} and LEMO \citep{LEMO}.  

A number of online annotation systems have emerged out of these frameworks, the vast majority of which are concerned with creating and viewing annotations. Only a handful go beyond this functionality and deal with annotation management. Many of these front-end systems have been analysed and compared extensively by Kahan et al. \citep{kahan2002annotea} and Haslhofer et al. \citep{LEMO}. To avoid exhaustive repetition, only those systems that are web-based and that include some functionality to manage or process annotations that have already been made, will be discussed here.

Amaya \citep{Amaya} is W3C's test-bed web editor/browser that includes an implementation of Annotea, a collaborative annotation system. Annotea/Amaya allows users to make and view annotations on a webpage. Some extra functionality is provided via a drop-down menu which allows users to reply to existing annotations, or to delete them \citep{Annotea}. Mozilla's instance of Annotea, Annozilla \citep{Annozilla}, provides the same options. Beyond merely creating and viewing annotations, users can also reply to them and delete them. 

This is very similar to the Google Drive "comments" system \citep{GDrive}. It is standard today for systems that integrate annotations to allow users to write (and edit) view, reply to and delete annotations. Google Drive adds one more piece of functionality to this which is the ability to mark annotations as "resolved". This hides the annotations, but they can still be viewed in the document history, and restored if need be.

A.nnotate \citep{AnnotateCom} (the software that Siyavula currently uses to annotate PDF books) offers users some processing of annotations or "notes". Apart from browsing annotations one page at a time in the context of the PDF, users can also view all notes made on a PDF. They can then sort the annotations displayed by date, subject, tag and document. It is possible to search for text in annotations, and to filter by tag and a few other predefined options such as "Include all notes/notes on text/notes on images". In addition, users can export annotations as CSV files.

The Open Knowledge Foundation's Annotator (upon which Siyavula's new annotation software is based) \citep{Annotator} offers a simple and user-friendly front-end system which allows users create annotations on any website. Although it comes packaged with a hosted web service for storing annotations (AnnotateIt \citep{AnnotateIt}) or a customisable storage API \citep{AnnotatorAPI}, neither of these options provides functionality for processing existing annotations.

The Debora (Digital access to Books of the Renaissance) interface \citep{debora} is unfortunately no longer functional online \citep{DeboraLink}. The original interface did go beyond merely displaying annotations: it allowed users to "chain together" paths of annotations, and then group those in "virtual chapters" and "virtual books" \citep{debora}. They did so to help users navigate between different content and annotations. This is surely one of the earliest online examples of an annotation interface giving users additional power to manipulate existing annotations. 

Mojiti \citep{Mojiti} is also no longer available online but can be accessed via the Internet Archive \citep{InternetArchive}. It allowed users to annotate videos online. Beyond this, it also allowed users to share their annotated videos by sending a link to the data or embedding it. This service has arguable been replaced by Google's YouTube Video Annotations \citep{YouTubeAnns}, which provides the same kind of functionality today. Users can make and view annotations in video content, but YouTube provides no annotation-specific functionality beyond this. 

Vannotea \citep{Vannotea} is annotation software for audiovisual content, built upon Annotea. It provides users with basic search and filter functionality of existing annotations. Users can filter annotations based on their associated metadata (e.g. author or date), and also perform a basic search for keywords in one or more metadata fields \citep{AnnoteaSidebarDoc}. Additionally, Vannotea provides a timeline that corresponds to the length of the audiovisual material, and Annotea annotations are surfaced (as icons) on this, allowing for easy visual browsing. 

Like Annotator, the Mozilla Firefox plugin WebAnnotator \citep{WebAnnotator1} \citep{WebAnnotator2} allows for easy annotating of the web. However, the only extra functionality it provides beyond making annotations is the option to save and export them. 

The current version of Yawas \citep{YawasLink} allows users to highlight content in a webpage, which is then stored as Google Bookmarks \citep{GBookmarks}. Users can add tags to saved bookmarks and can search for them as they can for any saved Google bookmark. The original version of Yawas \citep{Yawas} also allowed users to search for existing annotations and to import and export them. 

One noteworthy project that will enter (and possibly dominate) the online annotation space soon is Hypothes.is \citep{Hypothesis}. The beta version of the Hypothesis annotation system looks very promising, and in the near future this software may well present a solution to the problems involved with annotating the Web. However it is not yet clear whether it will offer back-end processing functionality, or merely a very elegant solution to creating and viewing annotations online.

Whilst there are many interfaces that enable users to make and manage annotations, it is evident that only a handful provide extended functionality beyond simply making, viewing and deleting annotations. Pieces of the web annotation problem have been solved in different projects, but there is not one system that presents a comprehensive solution. For example, the interface to make, view and edit annotation-like comments in Google Drive is hugely successful, but it is not coupled with comment filtering functionality. A.nnotate provides better filtering and sorting capabilities than most, but it is limited to PDF documents and can not handle HTML documents. There also exists no framework to address the post-processing of annotations, or any significant research into how databases of stored annotations can and should be accessed and manipulated. 

It is apparent that further research needs to be done into emerging (and increasingly complex) workflows involving existing annotations. It is probable that Siyavula's "annotation processing" workflow involves new use cases for annotations, and that there are others like it that are undocumented. It is also apparent that a research opportunity exists around new functionality and designs specifically tailored for annotation management. It is time to begin thinking (and developing) beyond how annotations are made, and to start exploring what can be done with annotations that already exist. 
