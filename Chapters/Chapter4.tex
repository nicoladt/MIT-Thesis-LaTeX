% Chapter 4

\chapter{Methodology} % Main chapter title

\label{Methodology} % For referencing the chapter elsewhere, use \ref{Chapter1} 

\lhead{Chapter 4. \emph{Methodology}} % This is for the header on each page - perhaps a shortened title

%----------------------------------------------------------------------------------------
\section{Introduction}
In order to design a prototype interface that would meet users' needs and solve the problem of being able to interact with and manipulate sets of annotations, a user-centred design process was undertaken. This included establishing user requirements; analysing those requirements to understand job roles and to build a conceptual model of the system; and researching design principles and guidelines to inform the development process. Participatory design sessions were held in order to create a paper prototype of the interface. This was converted into a high-fidelity prototype, which was formatively evaluated via usability tests. Improvements were made to the interface based on this evaluation, and then summative evaluation took place to determine whether or not the finished product met the original requirements and could be deemed a success. 

\section{User-centered design}
User-centered design (UCD) is a design philosophy and framework that aims to include end-users in as many stages of the design process as possible \citep{Abras}. The purpose of this inclusion is to use real user tasks and goals to drive development in order to design systems that are relevant to users and that adequately address their needs \citep[p. 327]{RogersPreece}.

Gould and Lewis \citep{GouldLewis} outline three principles which are now standard tenets of the user-centered approach, namely: early focus on users and tasks, empirical measurement and iterative design \citep[p. 327]{RogersPreece}.
 
Team participation (in planning, brainstorming and development) is a core part of Siyavula's corporate culture, and because the system was being designed for a niche group of users, a heavily user-centered process was deemed particularly appropriate in this context.

The UCD framework provides tools and guidelines for design but it is not prescriptive - design and process requirements are dependent on user input. Each design process is therefore potentially unique and may require a different approach or set of interventions \citep[p. 332]{RogersPreece}. In this instance, a meta-process was outlined before research commenced (establish requirements, design a low fidelity prototype, develop this into a high fidelity prototype, iteratively test) but the details of each stage of were left open-ended. This meant that the process could evolve organically from user input and could therefore be individually tailored to a specific problem and group of users. For example, although a focus group wasn't initially considered, it emerged from the initial requirements gathering phase that it would be a useful way to resolve conflicting user statements. Similarly the development and evaluation process was flexible and iterative - issues identified during development and testing had an effect on the final high-fidelity prototype. 

Although it is a flexible framework, UCD includes several well-established methodologies designed to capture the above-mentioned principles in the development process. These include establishing and analysing user requirements (e.g. by conducting user interviews, observing existing workflows and holding focus groups), prototyping (which could take the form of conceptual and/or participatory design) and iterative evaluation (and improvements to the design), during which users are asked to test and provide feedback about the ongoing product \citep[p. 330 - 331]{RogersPreece}. The methodologies mentioned here are a subset of tools that proved to be particularly appropriate and useful for the design process in question.

Questions that needed to be answered before a new interface could be designed included: 
\begin{itemize}
 \item How do different internal users need to interact with existing annotations? 
 \item What new functionality do users require? 
 \item How would users want and expect the interface to look and behave?
\end{itemize}

In order to answer these questions and to contextualise the problem, the design process began with establishing user requirements. 
\section{Establishing requirements}
 

\subsection{Stakeholder identification}
The stakeholders were divided into three broad categories of job role: "Development", "Production" and "Sales". These three categories corresponded to three different internal teams of employees, but were also generally correlated to computer experience levels. Whilst all employees were fully computer literate, those in the "Development" category were the most advanced users (professional programmers and command line experts). "Production team" users had above-average (for the team) computer skills (basic programming, familiarity with markup languages like XML and LaTeX) and "Sales" users had average computer skills, being fully proficient with computers and the Internet,  but unfamiliar with command line operations and programming or markup languages.

\subsection{Stakeholder interviews}
Traditional contextual interviews (including observations) were dismissed as a viable option because the annotation software was still being developed and there was no existing workflow to observe \citep[p. 38]{BeyerHoltzblatt}. Alternative annotation systems used by employees (such as a.nnotate.com) were markedly different to the system being developed (they also had no back-end mechanism by which users could process annotations) so there was no analogous, precedent workflow that could be used in a contextual interview. 

Instead, stakeholder identification and one-on-one stakeholder interviews were held in order to identify user requirements. Because of the small group of users involved, interviews were a practical means of getting individual feedback from every possible user of the system. A fairly open-ended interview process allowed for individual discussion and discovery about the varied ways in which different team members would interact with the system. 

\subsection{Focus group}
Once the interviews were complete, an informal focus group was held with all users to clarify points raised in the interviews and finalise the user requirements identified in the requirements analysis process \citep[p. 365]{RogersPreece}. 

The focus group enabled users to exchange and clarify ideas and issues arising from the interviews and also collectively reach consensus on their requirements. As well as being a mechanism to obtain detailed user input, interviews and brainstorming allowed users to feel that they were participating actively in the design process, to voice any concerns and express their wishes for the future annotation system \citep[p. 365]{RogersPreece}. 

\subsection{User requirements analysis}
During stakeholder interviews and in the focus group, individuals were asked the general question "Why annotate?". Several answers emerged, including: 
\begin{itemize}
 \item To improve existing products (post-publication)
 \item To collaboratively build new products (pre-publication)
 \item To involve community at all stages of product development
\end{itemize}

Stakeholders agreed that annotations must have a type, an associated priority (by date – e.g. old errata are most important) and status (e.g. new, open or resolved). Discussion and replies on a resolved or processed annotation should not be possible - either these annotations must be hidden, or locked down. 

Annotations should be assignable to team members (e.g. certain chapters could be pre-assigned to certain team members) and re-assignable (e.g. for specific questions, or to get a second team member to double check a processed annotation before it is marked as resolved).

Stakeholders agreed that annotations should be sortable (by date, type, location, status, priority, user) and browsable (by subject, book, chapter etc.) and that it should be possible to build a query based on the latter categories.

Some stakeholders suggested that push notifications via email would be useful. Internally, these could notify team members if there was a new annotation in one of their pre-assigned chapters, for example. Externally these could notify an external user if there was a reply to their annotation – e.g. requesting clarification; or to notify them that it had been resolved.

In terms of viewing existing annotations, stakeholders said that they would like to be able to view all information associated with an annotation, including location in book, highlighted text, username and user comment. They also agreed it would be useful to view an annotation in the context of the webbook in which it was made (to see the highlighted text); to be able to move easily from one annotation to the next; and to be able to easily view all annotations made in a given section of a webbook.

\subsection{Use cases}
The three job role categories mentioned above also generally correlate to three different types of use case, which emerged from the stakeholder interviews. "Development" users are seldom involved in the "processing" of annotations. They tend to work with the actual development and maintenance of the annotator software. "Production" users are typically involved in the bulk of annotation processing. They would be the most frequent and intense users of the new system, needing to do batch processing (viewing, searching, filtering etc.) per subject or per chapter. "Sales" users are less intensive users who typically need to locate a particular annotation made  by a specific volunteer, in order to provide feedback to that volunteer on the status of an annotation (e.g. "\textit{has an error been fixed or a suggestion implemented?}"). These use cases overlap in a complementary manner. "Sales" requirements are a subset of "Production" requirements, and "Developer" users could slot into either of these use cases if and when the need arises. 

\section{Conceptual model}
From the the stakeholder interviews and focus group a conceptual model of the interface emerged. Johnson and Henderson \citep[p. 27]{Johnson} describe a conceptual model as a "high-level description of how a system is organised or operates". Conceptual models provide metaphors and analogies that convey understanding about what a system is for and how to use it. They also include the concepts that users will encounter when using the system, the relationships between said concepts, and information about the mappings between the concepts and the task domain the system is supposed to support \citep[p. 40-41]{RogersPreece}.  The benefits of constructing a conceptual model before design begins include a simpler, more coherent end-product, and a better match between user expectations and design intentions \citep[p. 26]{Johnson}. Conceptual models can also be used to guide and inform the design process, keeping it as close to the original task requirements and user domain space as possible \citep[p. 40]{RogersPreece}. 

Using Johnson and Henderson's task-based conceptual model \citep[p. 30]{Johnson}, it was possible to view each annotation as an object with attributes. Attributes for each would include subject, grade, chapter number, highlighted text, username, user comment and a timestamp. Tasks would include grouping objects by attributes (e.g. annotations can be grouped per grade or subject) and combinations thereof. Other tasks or actions to be performed on annotation objects would include filtering by attribute, sorting by attribute and searching by username. 

In terms of metaphors and real-world analogies, searching for a particular item via a process of refinement occurs in many places in reality and on the Internet. The binary search (and example of a 'divide and conquer' algorithm) \citep[p. 48]{Neapolitan} performed to locate a word in a hard copy dictionary or telephone directory is a good analogue example of this. Online, all users are familiar with searching via refinement from Gmail (e.g. filter by label, or search inbox for a particular sender), GitHub (e.g. filter issues by label), Google, Flickr and Wikipedia, to name but a few. Users with programming skills were also familiar with the design of various kinds of sorting algorithms. 

Several annotation attributes (by which annotations could also be categorised) map directly to print book schema: subject, grade and chapter are a common hierarchy used in high school textbook publishing. Annotations themselves could be viewed as being Post-it notes, stuck into a particular book, about a particular section or page of text. Those Post-it notes could then be categorised, depending on where and when they were made - much like a paper library indexing system. 

\section{Design guidelines}
In addition to the conceptual model, standard design principles, guidelines and rules were also used to inform the design process, as well as standard interface design and behavioural patterns. Design principles represent a mixture of theoretical knowledge, practical experience and common sense \citep[p. 26]{RogersPreece}. Although design principles are usually fairly high-level and abstract they provide well-established suggestions for designers about best practices to follow and include in their work. 

The principles and rules used to guide the current design process are documented in depth in the literature; however a brief summary is provided below. These principles were used in this research to guide all design and development in a reflective manner. When migrating from the paper-prototype to high-fidelity prototype it was helpful to interrogate whether or not each new interface element being developed still conformed to best practices. During usability testing, if users gave conflicting feedback, design guidelines were also used to fine a best possible solution. For example, if catering to one user's expectations would mean overriding functionality that another user particularly approved of, then the opinion that best fit good design principles was favoured. 

Rogers, Sharp and Preece \citep[p.~26-29]{RogersPreece} suggest five design principles to follow, namely:
\begin{itemize}
 \item  \textbf{visibility}: the more visible functionality is, the more likely it is that users will know what to do next.
\item \textbf{feedback}: giving users feedback about what they have done or what has happened allows them to continue with the activity.
\item \textbf{constraints}: restricting the kinds of interaction that can take place at any point in time.
\item \textbf{consistency}: design interfaces to have similar operations and use similar elements for achieving similar tasks.
\item \textbf{affordance}: give objects attributes in such a way that users will know how to use the object.
\end{itemize}
Similarly Dix and Finlay \citep[p. 260]{DixFinlay} outline: 
\begin{itemize}
 \item \textbf{learnability}: the ease with which new users can start interacting with a system and know how to use it (including predictability, synthesizability, familiarity, generalisability and consistency). 
 \item  \textbf{flexibility}: how many ways in which a user and a system can effectively exchange information (including dialog initiative, multi-threading, task migratibility, substitutivity and customizability).
 \item  \textbf{robustness}: the level of support provided to the user in order for them to successfully assess and achieve goals (including observability, recoverability, responsiveness and task conformance). 
\end{itemize}

No design work can be undertaken without acknowledgement of Ben Shneiderman's \textit{Eight Golden Rules of Interface Design} \citep{ShneidermanPlaisant}:
\begin{enumerate}
 \item Strive for consistency.
 \item  Cater to universal usability.
 \item Offer informative feedback.
 \item Design dialogs to yield closure.
 \item Prevent errors.
 \item Permit easy reversal of actions.
 \item Support internal locus of control.
 \item Reduce short-term memory load.
\end{enumerate}
Similarly, Nielsen and Molich's \textit{10 Usability Heuristics for User Interface Design} \citep[p. 249]{NielsenHeuristics} were included in the process (see  Table \ref{table:heuristics}).

\section{Participatory design}
Once the requirements gathering and analysis process was complete and a conceptual framework and design guidelines were in place, participatory prototyping sessions could be undertaken, to integrate as much user input as early on in the design process as possible. 

The participatory design model provides a framework in which designers and users (and any other stakeholders) can collaborate during the design process. It allows for ongoing user input and the inclusion of user expertise and knowledge. Prototyping is a participatory design method by which designers and users can research, explore and iteratively design together \citep{Spinuzzi}.   

Paper prototyping was chosen because it allowed for creative user involvement and an opportunity to build on the research obtained through the stakeholder interviews. Prototyping (and the dialogue that it involved) gave participants the opportunity to refine their initial ideas and to move from hypothetical discussion and conceptual design about their future use of the system to concrete functionality and paper mockups. It was also selected because it allows for easy visualisation and exploration of new interface features, using tools (papers and coloured pens) that all participants are familiar with \citep[p. 380]{HackosRedish}. 

Ongoing user input was deemed particularly valuable given that there existed no precedent for the back-end interface in question, and that future users had rather specialised user requirements for the interface. Additionally, all stakeholders (even those in the "Sales" category) were computer literate and familiar with a variety of web-based software, and therefore able to offer informed opinions about their work processes and needs. 

Two 1.5 hour paper prototyping sessions were held with a representative sample of team members, who had different job roles and work experience, and could therefore bring a variety of input and opinions to the design. The first session focused only on how users would expect to be able to interact with, search or filter annotations (not on how those annotations or search results would be displayed). The second session focused on the displaying of annotations once a user had implemented a search or set of filters. 

The aim of these sessions was to produce a paper prototype of the interface design that included new functionality for filtering, sorting and searching for annotations, and an interface for displaying filtered results. 

Once a paper-prototype existed, this could then be developed into a high-fidelity prototype which could be iteratively evaluated as discussed in the following section. 

\section{Evaluation}
According to Rogers, Sharpe and Preece \citep[p. 433]{RogersPreece} evaluation is a fundamental component of the design process. It allows designers to collect information about what users experience when interacting with a prototype. Evaluation focuses both on the usability and user experience of a prototype, and its purpose is to improve a prototype design.
\subsection{Usability testing}
One form of evaluation is usability testing. Usability tests involve collecting data using a variety of methods including observations, interviews and questionnaires. Rogers et al. \citep[p. 438]{RogersPreece}  state that the fundamental goal of usability tests "is to determine whether an interface is usable by the intended user population to carry out the tasks for which it as designed. This involves investigating how typical users perform on typical tasks." Similarly, Shneiderman and Plaisant \citep[p. 144]{ShneidermanPlaisant} state that "usability tests are designed to find flaws in user interfaces". According to Beyer et al. \citep[p. 373]{BeyerHoltzblatt} they "tune an interface at the tail end of design, to clean up any rough edges or unnecessary difficulty in understanding or interacting with the interface."

While usability tests can be performed in controlled laboratory settings (e.g. if the performance of a prototype needs to be measured), it is also possible to perform them in more casual settings familiar to the user \citep[p. 438]{RogersPreece}. This latter option was selected because the testing related to functionality and behaviour (not computing performance) and because of the simple practicalities involved in testing with users in their own workplace. Physical and system variables were kept consistent wherever possible and all tests used the same equipment and software. 

To evaluate the high-fidelity prototype interface in question, two rounds of usability tests were undertaken. The first set of tests was intended as formative evaluation, i.e. they were performed during the design process to ensure that the design was still conforming to user expectations and requirements \citep[p. 437]{RogersPreece} and design guidelines. The results from these tests were then analysed and used to improve the design of the prototype. A second round of summative usability testing was then undertaken, to assess the interface as a finished product. 

The usability tests were designed to include interview-style questions about what users thought aspects of the interface represented, and how they expected them to behave (without interacting with them). Additionally users were asked to perform a number of simple tasks, carefully selected and designed to evaluate various behavioural aspects of the interface. For the summative set of tests, users were also asked to complete a satisfaction questionnaire. 

Both sets of usability tests are discussed in detail in Chapter 7 however to summarise the process briefly: five new users were chosen for each set of tests and under the same conditions (in the boardroom at their office) users were asked the pre-determined questions and to complete the list of tasks. Users were tested in two groups of five because there were ten new users available from the team, who had not seen the interface before. 

Testing with these ten users (chosen to be a representative sample of the three user groups identified who would use the final interface \citep[p. 461]{RogersPreece}) as well as designing with the original four users meant that the entire group of users available to this process would have participated by the time summative evaluation was complete. Small sample sizes are common in usability testing, and it is well documented that a high number of usability issues can be determined from tests with just a few users \citep[p. 119]{HackosRedish}.

Each session was recorded using screencast software \citep{Kazam} that recorded the active desktop and the conversation. Additionally the evaluator took notes detailing user comments and actions. 

The data collected from each session was analysed for misconceptions, misunderstandings and misclicks. The aim of the analysis was to identify any interface elements or behaviours that confused users, or behaved differently to their expectations, and to "identify user interaction components or features that both support and detract from user task performance" \citep{GabbardHix}.

The results of the formative usability tests were summarised and used to inform design changes and improvements to the interface, to bring it more closely in line with user expectation and requirements. These improvements were then tested again. Results from the summative usability tests were used to establish whether or not the final interface adequately met user requirements and provided users with the new functionality they needed.

\subsection{QUIS questionnaire}
Users who participated in the summative usability tests were also asked to fill out several sections of the "Questionnaire for User Interaction Satisfaction" version 7 (QUIS7), to assess their satisfaction levels with the prototype interface \citep{QUIS}. QUIS is a usability tool, the reliability and validity of which is well documented \citep{Harper1997}. The questionnaire includes a demographic section, an overall measure of satisfaction, and measures of user satisfaction in four specific interface aspects including screen factors, terminology and system feedback, learning factors, and system capabilities \citep{Harper1993}.  

User responses to the questionnaire were analysed qualitatively to determine overall user satisfaction with the interface. 

\subsection{Heuristic evaluation}
To conclude the summative evaluation, a heuristic evaluation of the interface was performed once usability testing was complete. 

Heuristic evaluation is a usability inspection method used to evaluate user interfaces against a standardised set of usability principles called heuristics \citep[p. 506]{RogersPreece}. These principles were developed by Jakob Nielsen and colleagues \citep{NielsenMolich} and have been refined and reduced into a set of ten guidelines as listed below. 

A heuristic evaluation is performed by assessing how an interface conforms to and complies with such guidelines or heuristics. While it has been shown that increasing the number evaluators also increases the number of problems identified (Nielsen suggests that a minimum of three evaluators is ideal) there is arguably still merit in such an evaluation being performed by one individual \citep{NielsenHow}. The outcome of a heuristic evaluation should be a list of usability problems with the interface, with details about how they violated the guiding principles in the opinion of the evaluator \citep{NielsenHow}. 

The latest list of usability heuristics defined by Jakob Nielsen and colleagues \citep{NielsenMack} is listed in the following table. 

\begin{longtable}{|p{0.4cm} | p{5cm}|p{8cm}|}

 \hline
1. & Visibility of system status & The system should always keep users informed about what is going on, through appropriate feedback within reasonable time. \\ \hline

2. &Match between system and the real world & The system should speak the users' language, with words, phrases and concepts familiar to the user, rather than system-oriented terms. Follow real-world conventions, making information appear in a natural and logical order. \\ \hline

3. &User control and freedom& Users often choose system functions by mistake and will need a clearly marked "emergency exit" to leave the unwanted state without having to go through an extended dialogue. Support undo and redo.\\ \hline

4.& Consistency and standards & Users should not have to wonder whether different words, situations, or actions mean the same thing. Follow platform conventions. \\ \hline

5.& Error prevention & Even better than good error messages is a careful design which prevents a problem from occurring in the first place. Either eliminate error-prone conditions or check for them and present users with a confirmation option before they commit to the action.\\ \hline

6.& Recognition rather than recall & Minimize the user's memory load by making objects, actions, and options visible. The user should not have to remember information from one part of the dialogue to another. Instructions for use of the system should be visible or easily retrievable whenever appropriate.\\ \hline


7.& Flexibility and efficiency of use & Accelerators -- unseen by the novice user -- may often speed up the interaction for the expert user such that the system can cater to both inexperienced and experienced users. Allow users to tailor frequent actions.\\ \hline

8.&Aesthetic and minimalist design & Dialogues should not contain information which is irrelevant or rarely needed. Every extra unit of information in a dialogue competes with the relevant units of information and diminishes their relative visibility.\\ \hline

9.& Help users recognize, diagnose, and recover from errors & Error messages should be expressed in plain language (no codes), precisely indicate the problem, and constructively suggest a solution.\\ \hline

10.& Help and documentation & Even though it is better if the system can be used without documentation, it may be necessary to provide help and documentation. Any such information should be easy to search, focused on the user's task, list concrete steps to be carried out, and not be too large. \\
\hline

\caption{Nielsen's 10 usability heuristics}
\label{table:heuristics}
\end{longtable}

The combination of summative usability tests, a user satisfaction questionnaire and a heuristic evaluation allowed for a thorough assessment of the success of the interface in meeting users requirements and therefore the success of the interface in solving the larger annotation-processing problem that Siyavula faced. These complementary evaluation methods not only captured great detail about user behaviour and expectations (e.g. in the usability tests) but also included higher-level feedback and analysis from the questionnaire and the heuristic evaluation.
